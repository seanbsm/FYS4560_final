\documentclass[11pt,a4paper]{article}

\usepackage{amsmath}
\usepackage{amssymb}
\usepackage[english]{babel}
\usepackage[utf8]{inputenc}
\usepackage{multicol}
\usepackage[cm]{fullpage}
\usepackage{comment}
\usepackage[pdftex]{graphicx}
%\usepackage{GFS artemisia}
%\usepackage{verbatim}
\usepackage{tikz}
\usepackage{listings}
\usepackage{bm}
%\usepackage{sbbm}
\usepackage{float}
\usepackage{mathtools}
%\usepackage{subfigure}
\usepackage{fancyhdr}
\usepackage{simplewick}
\usepackage{slashed}
\usepackage{xfrac}
%\usepackage{tikz-feynman}

\usepackage[pass]{geometry}
%\usepackage{graphicx}
\usepackage{caption}
\usepackage{subcaption}

% % % Pagestyling
\pagestyle{fancy}
\fancyhf{}
\headsep = 25pt
% % %


% % % New commands
\newcommand{\HRule}{\rule{\linewidth}{0.5mm}}
\newcommand{\down}{\Big|\frac{1}{2}\:-\frac{1}{2}\Big\rangle}
\newcommand{\up}{\Big|\frac{1}{2}\:\:\:\:\frac{1}{2}\Big\rangle}
\renewcommand{\arraystretch}{1.5}
\newcommand{\bpm}{\begin{pmatrix}}
\newcommand{\epm}{\end{pmatrix}}

% % % Nice column vectors
\makeatletter
\newcommand{\Spvek}[2][r]{%
	\gdef\@VORNE{1}
	\left(\hskip-\arraycolsep%
	\begin{array}{#1}\vekSp@lten{#2}\end{array}%
	\hskip-\arraycolsep\right)}

\def\vekSp@lten#1{\xvekSp@lten#1;vekL@stLine;}
\def\vekL@stLine{vekL@stLine}
\def\xvekSp@lten#1;{\def\temp{#1}%
	\ifx\temp\vekL@stLine
	\else
	\ifnum\@VORNE=1\gdef\@VORNE{0}
	\else\@arraycr\fi%
	#1%
	\expandafter\xvekSp@lten
	\fi}
\makeatother
% % %

% % %

\renewcommand*\thesubfigure{\arabic{subfigure}}
\renewcommand{\labelenumii}{\theenumii}
\renewcommand{\theenumii}{\arabic{enumii}.}
\renewcommand{\labelenumiii}{\theenumiii}
\renewcommand{\theenumiii}{\arabic{enumiii}.}

% % % Enables Matlab-scipt import
\lstset{
        language=Matlab,                            
        numbers=left,                               
        numberstyle=\footnotesize,                  
        stepnumber=1,                               
        numbersep=5pt,                              
        showspaces=false,                           
        showstringspaces=false,                     
        showtabs=false,                             
        breaklines=true,                            
        breakatwhitespace=false,
        frame=single,
        commentstyle=\color{green},
        keywordstyle=\color{blue},
        escapeinside={\%*}{*)}  }
% % %

\begin{document}

% % % Titlepage start
\begin{titlepage}
\begin{center}
\medskip
\textsc{\LARGE FYS4560;}\\[0.3cm]
\textsc{\Large Elementary Particle Physics}\\[1.5cm]

\textsc{\Large Final Project}\\[0.5cm]

\HRule \\[1.0cm]
\textmd{ \huge Higher Dimensions; \\ \LARGE Theoretical and Experimental Aspects \\[0.4cm] }
\HRule \huge \\[1.5cm]

\begin{minipage}{0.4\textwidth}
\begin{center}
\large\textsc{Sean B.S. Miller}\\

\end{center}
\end{minipage}

\vfill

\begin{abstract}
	The Randall-Sundrum (RS) model of compacted dimensions will be studied, together with the postulate of the graviton particle. Quark-anti-quark production of the simplest possible massive graviton (1st order tower of Kaluza-Klein excitations) will be calculated.
\end{abstract}

\vfill

{\large \today}

\end{center}
\end{titlepage}
% % % Titlepage end

\newpage

\fancyhead[L]{\textsc{Final Project}}
\fancyhead[C]{\textsc{\today}}
\fancyhead[R]{\textsc{Sean B.S. Miller}}
\fancyfoot[C]{\thepage}

\section{Introduction}
Higher dimensions, also called extra dimensions, are physical models for the dimensionality of our universe, mostly suggested with the goal of explaining the hierarchy problem of the standard model. There are \emph{many} theories for higher dimensions. The most recognised are:
\begin{itemize}
	\item \emph{Large extra dimensions}: The often-heard theory that gravity acts through several dimensions, therefore becoming weaker. It originates from the ADD model as an attempt to solve the hierarchy problem\footnote{The "hierarchy problem" is the problem in explaining why gravity and the weak force are so weak compared to QED and QCD.}.
	\item \emph{Warped extra dimensions}: Describing our universe as a five-dimensional anti-de Sitter space, and claiming the SM particles are localized on a (3 + 1)-dimensional brane(s).
	\item \emph{Universal extra dimensions}: All particles move universally through the extra dimensions, unlike to two other models where only gravity propagates through them.
\end{itemize}

Obviously, a thorough description of any of these models is near impossible for such a small paper, let alone all the models together. Therefore, a brief outline of the theory behind the two currently most promising\footnote{"Promising" in the sense that they provide measurable outcomes, and fit very well with what we already know from the standard model. The problem is that so far they have predicted nothing. Finding a graviton would possibly confirm either theory.} models will be given.\\
The first is the large extra dimension model by Arkani-Hamed, Dimopoulos, and Dvali (ADD). Originally, it was proposed as a model to explain the hierarchy problem (why the weak force is $10^{32}$ times stronger than gravity, among other problems). The extra dimensions\footnote{Today, 6 dimensions is the most common expansion.} are then suggested as planes into which gravity, assumed just as strong as the other forces, spreads. Therefore gravity becomes "diluted", while the known SM particles stay in (1,3)-spacetime.\\
The second model is the warped extra dimension model by Randall and Sundrum (RS), made due to disliking the current universal extra dimensions models. They assumed that, rather than having universal extra dimensions in which all particles propagate,  there is a small extra dimension. This means they model our world as a 5-dimensional anti-de Sitter space\footnote{This will be explained later on.}. By small, it means the extra dimension has a large curvature, or is \emph{warped}. From general relativity, gravity and curvature are very much the same thing, and therefore the extra dimension, called the Planckbrane, can easily host gravitons.\\
A question that then springs to mind is why exactly gravitons and extra dimensions are connected (other than gravitons "carrying" gravity). If the standard model is expanded, but without inclusion of extra dimensions, to include a graviton field, then measuring it would be, at best, very optimistic\footnote{Refer to "Can Gravitons be Detected?" by Rothman and Boughn (https://arxiv.org/pdf/gr-qc/0601043v3.pdf) for an impression of the problem.}.
Should any extra dimensional model be true, it would certainly be desirable to prove it by measurement. Finding a particle with the properties of the graviton would mean that at least \emph{some} extra dimensions model is true.

\section{Kaluza-Klein theory}
The reason Kaluza-Klein theory is discussed is because one requires knowledge of the so-called "Kaluza-Klein towers"; massive excitations of an expansion model\footnote{This expansion depends on the model, and is where RS and ADD differ.} of the spacetime metric.\\
As already mentioned, one, or several, extra spatial dimension(s) is assumed, but only one extra will be considered for the sake of simplicity. This means that the spacetime metric goes "up in size", i.e.

\begin{equation}
	\tilde{g}_{ab}(x,\theta) ) = 
	\begin{pmatrix}
		g_{\mu\nu}(x,\theta)-B_\mu(x,\theta) B_\nu(x,\theta) \Phi(x,\theta) & B_\mu(x,\theta)\Phi(x,\theta)\\
		B_\nu(x,\theta)\Phi(x,\theta) & -\Phi(x,\theta)
	\end{pmatrix}
\end{equation}

where
\begin{itemize}
	\item Latin indices represent all dimensions, greek indices represent 4-dimensional spacetime.
	\item $g_{\mu\nu}$ is the 4-dimensional spacetime metric from before.
	\item $B_\mu$ is a gauge field, defined by $B_\mu \equiv \frac{g_{\mu5}}{g_{55}}$
	\item $\Phi$ is a scalar field, defined by $\Phi = g_{55}$.
	\item $x$ are the normal spacetime coordinates, while $\theta$ is the extra coordinate. The notation is due to the extra dimension(s) often being "circular" (such as the $S^1$ space).
\end{itemize}

As can be seen, a complete description depends on $g_{\mu5}$ and $g_{55}$. However, this is precisely where the complication of extra dimensions arises. The metric depends on the space, and the shape of the space is what we do not know\footnote{This is why the field of extra dimensions can be so large and complex, as there are \emph{many} ways to guess at a shape of the new dimension(s), and each model gives a characteristic outcome which must fit experiments at, for example, the LHC.}. \\
Firstly, consider a simple model with a massive scalar field $\hat{\phi}(x,\theta)$,

\begin{equation}
	S_5 = \int dx^4 d\theta \sqrt{-\tilde{G}}\left(-\frac{1}{2}(\partial_ a\hat{\phi})(\partial^a\hat{\phi})^\dagger - \frac{m^2}{2}\hat{\phi}^2 - \frac{c_5}{4!}\hat{\phi}^4\right)
	\label{eq:KKaction}
\end{equation}

where $\tilde{G} \equiv \det(\tilde{g}_{ab})$ and $c_5$ is some coupling constant we do not yet know. The scalar field can be Fourier expanded as

\begin{equation}
	\hat{\phi}(x,\theta) = \sum_{\substack{n=\{n_1,n_2,\ldots,n_5\}; \\ n_i\in\mathbb{Z}\:\forall\:i}} \phi{(n)}(x)Y_n(\theta)
\end{equation}

where $Y_n(\theta)$ are orthogonal, normalized eigenfuntions of the Laplace operator on the internal space\footnote{The extra dimension/space, denoted $K_d$.}:

\begin{equation}
	\Delta_{K_d}Y_n(\theta) = \frac{\lambda_n}{R^2}Y_n(\theta)
\end{equation}

where $R$ is the "characteristic size" of the space $K_d$ (obviously a radius in the case of circular objects). Inserting this into equation \ref{eq:KKaction} and performing the integral over $K_d$ gives

\begin{align}
	\begin{split}
	S_5 = \int dx^4 \sqrt{-\tilde{G}}\bigg(&-\frac{1}{2}(\partial_ \mu\phi^{(0)})(\partial^\mu\phi^{(0)})^\dagger - \frac{m^2}{2}\left(\phi^{(0)}\right)^2\\
	& - \sum_{n\neq0} \left[(\partial_ \mu\phi^{(n)})(\partial^\mu\phi^{(n)})^\dagger + m_n^2\phi^{(n)}\phi^{(n)\dagger}\right]\\
	& - \frac{c_5}{4!}\left(\phi^{(0)}\right)^4 - \frac{c_5}{4}\left(\phi^{(0)}\right)^2\sum_{n\neq0} \phi^{(n)}\phi^{(n)\dagger} - \ldots\bigg),
	\end{split}
\end{align}

and by comparison one sees $m_n^2 = m^2 + \frac{\lambda_n}{R^2}$. The dots represent all terms without any $\phi^{(0)}$-coefficients. As one sees, there is an infinite number of masses, since there is an infinite number of eigenvalues; a mathematical problem without any current answer. The hope is that these modes\footnote{The infinite set of which is called the Kaluza-Klein tower of modes. (Often one hears of KK towers.)} can be tied to the graviton, and will be explained better in the following sections.

\section{The Arkani-Hamed-Dimopoulos-Dvali model}
In the ADD model, the standard metric is expanded upon by introducing a field $\hat{h}_{ab}(x,\theta)$ such that:

\begin{equation}
	\tilde{g}_{ab}(x,\theta) = \eta_{ab} + \frac{2}{M^{1+d/2}}\hat{h}_{ab}(x,\theta)
\end{equation}

\section{The Randall-Sundrum model}


\section{Graviton production at the LHC}


\section{A simulated approach}


\section{Conclusions}


\end{document}